\begin{abstract}
\thispagestyle{plain}
As fossil fuels across the world are steadily depleted, the majority of the
world's energy production will shift towards renewable sources. However,
renewable energy sources do not always guarantee the same reliability in rates
of production. Therefore, new strategies must be developed to match supply and
demand in the face of unpredictable supply. The growing popularity and
proliferation of smart grid technology has lead to it being considered as one
possible method of solving this problem.

In this paper, a network implementation of a game theoretic approach to this
problem is formulated in a smart nanogrid to investigate firstly whether or not such an
approach is feasible in a network, and secondly whether or not it would
provide a better approach to the problem. A number of optimisation techniques,
such as convex optimisation and hyperplane projection optimisation, are also
employed to find a better solution. The goal of this project is to find an
implementation that finds a correct solution to the problem quickly and efficiently.

Ultimately, the method is found to be suitable only in situations involving a
small number of devices within a system. This is due to a problem with
scalability, where the system is slowed down by a large number of network
communications in order to find consensus between the devices taking part in the
system.
\end{abstract}
