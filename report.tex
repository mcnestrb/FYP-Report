% Created 2017-04-17 Mon 18:11
% Intended LaTeX compiler: xelatex
\documentclass[a4paper, notitlepage]{report}
\usepackage{graphicx}
\usepackage{grffile}
\usepackage{longtable}
\usepackage{wrapfig}
\usepackage{rotating}
\usepackage[normalem]{ulem}
\usepackage{amsmath}
\usepackage{textcomp}
\usepackage{amssymb}
\usepackage{capt-of}
\usepackage{hyperref}
% set the margins here, if they need to be modified
\usepackage[a4paper, left=1.3in, right=1.0in]{geometry}

% The linespacing you want
\linespread{1.4}

% Lets you pick fonts. Only works if you compile with xelatex or luatex
\usepackage[no-math]{fontspec}

% For block quotes
\usepackage{attrib}

% Additional mathsy symbols
\usepackage{amsmath}
\usepackage{amssymb}

% Used to make the captions on figures sans serif
\usepackage{caption}

% For including images
\usepackage{graphicx}

% For fancy code listings
\usepackage{listings}

\usepackage[style=ieee,backend=biber]{biblatex}

% For tables that span multiple pages elegantly
\usepackage{longtable}

% For APA-style citations

% For maintaining a list of abbreviations
\usepackage{nomencl}

% For drawing diagrams
\usepackage{tikz}

% These closely match the SCSS word template fonts
% Won't work unless you compile with xelatex or luatex
\setmainfont[Mapping=tex-text]{Times New Roman}
\setsansfont{Helvetica}
\setmonofont{Courier New}

\pagestyle{headings}

\hypersetup{
    colorlinks,
    citecolor=black,
    filecolor=black,
    linkcolor=black,
    urlcolor=black
}

% Settings for how you want your code listings to look
\lstset{
basicstyle=\ttfamily\scriptsize,
keywordstyle=\ttfamily,
numberstyle=\rmfamily\tiny,
numbers=left,
commentstyle=\sffamily,
breaklines=true,
frame=single,
stringstyle=\ttfamily,
identifierstyle=\bfseries,
lineskip=1mm
}

% Change these as appropriate, and they'll be filled in automatically on the
% cover page. You can also use them throughout the document, so as not to have
% to type them again all the time.
\newcommand \authorname{Brian McNestry}
\newcommand \authoremail{mcnestrb@tcd.ie}
\newcommand \supervisorname{Dr. Donal O'Mahony}
\newcommand \degreetitle{B.A. (Mod.) Integrated Computer Science}
\newcommand \projecttitle{Electricity Trading Between Smart Nano-Grids: \newline Matching Supply and Demand in the Face of Unpredictable Supply}

% All the commands are defined in this file
\newcommand \inserttitlepage{\thispagestyle{empty}
\begin{center}
{\sffamily
{\Large University of Dublin}

\vspace{10pt}

\includegraphics[scale=0.5]{tcd/trinitycollege.pdf}

\vspace{10pt}

{\Huge TRINITY COLLEGE}

\vspace{80pt}

\textbf{ \Large \emph \projecttitle}

\vspace{30pt}

\authorname

\degreetitle

Final Year Project May 2017

Supervisor: \supervisorname

\vspace{100pt}

\large{School of Computer Science and Statistics
\\$ $\\
O'Reilly Institute, Trinity College, Dublin 2, Ireland}
\linespread{1}
}
\end{center}

}
\newcommand \insertabstract{\begin{abstract}
\thispagestyle{plain}
As fossil fuels across the world are steadily depleted, the majority of the
world's energy production will shift towards renewable sources. However,
renewable energy sources do not always guarantee the same reliability in rates
of production. Therefore, new strategies must be developed to match supply and
demand in the face of unpredictable supply. The growing popularity and
proliferation of smart grid technology has lead to it being considered as one
possible method of solving this problem.

In this paper, a network implementation of a game theoretic approach to this
problem is formulated in a smart nanogrid to investigate firstly whether or not such an
approach is feasible in a network, and secondly whether or not it would
provide a better approach to the problem. A number of optimisation techniques,
such as convex optimisation and hyperplane projection optimisation, are also
employed to find a better solution. The goal of this project is to find an
implementation that finds a correct solution to the problem quickly and efficiently.

Ultimately, the method is found to be suitable only in situations involving a
small number of devices within a system. This is due to a problem with
scalability, where the system is slowed down by a large number of network
communications in order to find consensus between the devices taking part in the
system.
\end{abstract}
}
\newcommand \declaration{\include{tcd/declaration}}
\newcommand \acknowledgements{\chapter*{Acknowledgements}

\begin{itemize}
  \item Dr Donal O'Mahony for being an excellent supervisor
  \item Luiz Da Silva and Georgios Iosfidis for their help with helping me to understand
optimisation techniques
\end{itemize}}
\newcommand \permissiontolend{\include{tcd/permissiontolend}}

\newcommand{\argmax}[1]{\underset{#1}{\operatorname{arg}\,\operatorname{max}}\;}

\usepackage{datetime}

\def\fullhrulefill{\leavevmode\leaders\hrule height 1pt\hfill\kern 0pt}

\newcommand{\signedanddate} {
  \par\noindent\makebox[2.5in]{\fullhrulefill}
  \par\noindent\makebox[2.5in][l]{\authorname, May 5 2017}
}

\newcommand \needcite[1]{\underline{#1}}

\newcommand \abbrev[2]{#1\nomenclature{#1}{#2}}


% For changing the names of the List of Listings, etc.
\renewcommand*{\lstlistlistingname}{List of Listings}
\renewcommand*{\contentsname}{Table of Contents}
\renewcommand*{\nomname}{Abbreviations}

% Make the captions sans serif
\renewcommand{\captionfont}{\sffamily}

\author{Brian McNestry}
\date{\today}
\title{}
\hypersetup{
 pdfauthor={Brian McNestry},
 pdftitle={},
 pdfkeywords={},
 pdfsubject={},
 pdfcreator={Emacs 24.5.1 (Org mode 9.0.5)}, 
 pdflang={English}}
\begin{document}

\inserttitlepage

\pagenumbering{roman}

\declaration

\permissiontolend

\insertabstract

\acknowledgements

\tableofcontents

\newpage


\pagenumbering{arabic}

\part{Abstract}
\label{sec:org40b0344}

\part{Introduction}
\label{sec:org993cc07}


\part{Background}
\label{sec:org2b08cc9}
\chapter{Decentralised Grid}
\label{sec:org4c26c2c}
At present in Ireland and in many other countries, the national electric grid
infrastructure is controlled by a central body, namely the ESB. While there are
several electricity providers in Ireland, such as Bord Gáis Energy, SSE
Airtricity and Energy Ireland, each of them use the same distribution network as
one another. Essentially the power is provided from each of the different
providers and then routed into the same centralised hub belonging to the ESB.
From there, each consumer (a household) receives the energy that they pay for
accordingly at a fixed rate through that same infrastructure belonging to the
ESB. This is much the same system as any other country, where there is a
centralised grid. 

This system has been in place for decades and lends itself very well to the
situation where large companies can provide a steady supply of energy by way of
electricity plants that use both renewable and non-renewable energy sources.
Non-renewable energy sources, also known as fossil fuels, include resources such
as coal, gas and oil. While these are finite resources, at present they can be
burned at a steady rate in order to meet the demands of customers. Electricity
from renewable sources can also be produced at a fairly steady rate by placing
large farms in areas that are particularly well suited to the type of renewable
energy being produced. For example, large wind farms are set up in windy
regions far removed from residential or urban areas and solar panels can be
placed in regions that typically enjoy clearer skies than other areas.

However in the future, perhaps the very near future, with the ongoing depletion
of non-renewable resources, more and more people will turn to deploying solar
panels and local wind farms in their locale, regardless of whether or not they
are living in a particularly sunny or windy area.     
\chapter{Smart Grid}
\label{sec:org1094421}

\section{Nanogrids}
\label{sec:org9e07ad8}

\chapter{REFIT Scheme}
\label{sec:org1c01036}
\chapter{Auctions}
\label{sec:org27cdc33}


\chapter{Game Theory}
\label{sec:org45a398e}
\section{Cooperative Game Theory}
\label{sec:org80a932f}

\section{Non-Cooperative Game Theory}
\label{sec:org88a2d30}

\section{Cournot and Stackelberg Games}
\label{sec:org75e5f69}

\chapter{Optimisation Techniques}
\label{sec:orgb090d24}

\section{Convex Optimisation}
\label{sec:org6eb6923}

\section{Hyperplane Projection}
\label{sec:org2528aa4}

\part{Implementation}
\label{sec:orgf9f2f1e}

\chapter{Design}
\label{sec:orga7d36e1}

\chapter{Framework}
\label{sec:org0274879}

\chapter{Processes}
\label{sec:org56faaed}

\part{Conclusion}
\label{sec:org70d817a}

\chapter{Assessment}
\label{sec:orgc5a67ae}

\chapter{Future Work and Continuations}
\label{sec:org1282ad3}

\bibliography{bibliography}
\appendix
\end{document}